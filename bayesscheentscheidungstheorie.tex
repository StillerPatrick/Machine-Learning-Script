
\subsection{Definitionen}

Sei \begin{math} x \in X \end{math} eine Zufallsvariable und ein Element  aller Beobachtungen. Die Elemente der Menge aller Beobachtungen sind meistens kontinuierlich. \\ \newline 
Sei \begin{math} k \in K \end{math} eine Zufallsvariable und ein Element aller Klassen.
Die Elemente der Menge aller Klassen sind meistens diskret.\\ \newline
\textbf{Erkl\"arung} \\ \newline 
Da die Parameter der Wirklichkeit ,wie K\"orbergr\"oße , Gewicht usw.  meistens kontinuierliche Werte sind, ist es sinnvoll die Menge aller Beobachtungen kontinuierlich zu wählen, um dann diese diskreten Klassen zuzuordnen. Die Menge der Klassen ist meistens Diskret ,da wir in den meisten Fällen eine endliche Anzahl an Klassen haben.\\ \newline
\textbf{Wahrscheinlichkeiten}\\ \newline
\begin{math} P(k_{i}) \end{math} Ist die Wahrscheinlichkeit der Klasse $ k_{i}$ und wird auch a priori Wahrscheinlichkeit genannt.\\ \newline
$P(x|k_{i})$ ist die Wahrscheinlichkeit des Auftretens der Beobachtung x unter der  Bedingung der Klasse $k_{i}$ und wird auch klassenbedingte Wahrscheinlichkeit genannt.\\ \newline
$P(k_{i}|x)$ ist die Wahrscheinlichkeit der Zuordnung einer bestimmten Klasse unter der Bedingung der Beobachtung x und wird auch a posteriori Wahrscheinlichkeit genannt.
